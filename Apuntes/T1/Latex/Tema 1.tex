\documentclass[12pt,a4paper]{report}

% --- Paquetes de Idioma y Codificación ---
\usepackage[utf8]{inputenc}
\usepackage[T1]{fontenc}
\usepackage[spanish, es-tabla]{babel}

% --- Diseño de Página ---
\usepackage{geometry}
\geometry{top=2.5cm, bottom=2.5cm, left=3cm, right=3cm}

% --- Matemáticas y Símbolos ---
\usepackage{amsmath, amssymb, amsthm}

% --- Colores y Cajas ---
\usepackage[dvipsnames]{xcolor}
\usepackage[most]{tcolorbox}

% Caption

\usepackage{caption}

% Gráficos

\usepackage{graphicx}

% Espaciadora

\newcommand{\esp}{\vspace{0.5cm}}

% Bibliografía

\usepackage[style=apa, backend=biber]{biblatex} % Estilo APA
\addbibresource{biblio.bib} % Nombre de tu archivo .bib

% Configuración de cajas personalizadas
\newtcolorbox{importante}{
	colback=red!5!white,
	colframe=red!75!black,
	title=Importante,
	fonttitle=\bfseries
}

\newtcolorbox{nota}{
	colback=blue!5!white,
	colframe=blue!75!black,
	title=Nota,
	fonttitle=\bfseries
}

% Definición de la caja de bibliografía
\newtcolorbox{bibliografia}{
	colback=gray!5!white,
	colframe=gray!75!black,
	title=Bibliografías útiles,
	fonttitle=\bfseries,
	breakable
}

% Socrative

\newtcolorbox{Socrative}{
	colback=orange!5!white,
	colframe=orange!75!black,
	title=Socrative,
	fonttitle=\bfseries,
	breakable
}

% Float

\usepackage{float}

% --- Hipervínculos ---
\usepackage{hyperref}
\hypersetup{
	colorlinks=true,
	linkcolor=blue,
	filecolor=magenta,      
	urlcolor=cyan,
}

% --- Datos del Documento ---
\title{Óptica II}
\author{Chengyu Jin}
\date{\today}

\begin{document}
	
	\maketitle
	
	\tableofcontents
	\newpage
	
	\chapter{Teoría de la coherencia parcial de la luz. Aplicaciones.}
	
	\begin{bibliografia}
		\begin{enumerate}
			\item Nieves, J.L., Jiménez J.R. y Hernández Andrés, J. “Introducción a la teoría
			difraccional de la formación de imágenes”. Cap 5.
			\item  Pedrotti, F.L., Pedrotti, L.S. “Introduction to Optics”. Cap 12.
			\item Casas, J. “Óptica”. Cap. 12.
			\item Fowles, G.R. “Introduction to modern optics”. Cap. 3
			\item Born, M. y Wolf, E. “Principles of Optics”. Cap. 10.
			\item Hecht-Zajac.“Óptica”.Cap.12.
		\end{enumerate}
	\end{bibliografia}
	
	\section{Repaso de conceptos elementales y definiciones}
	
	\subsection{Coherencia temporal, tiempo de coherencia y longitud de coherencia.}
	
	Supongamos ahora que tenemos un interferómetro Michelson iluminado con una fuente casi puntual $\sigma$ que emite con un ancho de banda en frecuencias $\Delta\nu$. (\cite{Resumen})
	
	\esp
	
	\begin{nota}
		En la realidad no existen rayos monocromáticos debido al principio de incertidumbre de Heisenberg, los rayos siempre tienen un ancho de banda $\Delta\nu$
	\end{nota}
	
	\esp
	
	A la diferencia de recorrido máxima en distancias entre los dos haces para que haya interferencias,
	que se calcula como $c\Delta\tau$, denominamos \textbf{longitud de coherencia} de la fuente.
	
	\begin{equation}
		\Delta = 2(d_2-d_1)\cos\theta = c\Delta\tau = l_C
	\end{equation}
	
	Experimentalmente se ha determinado que para que existan interferencias necesitamos que $\Delta\nu\Delta\tau < 1$ esto es que $\Delta\tau = \tau_C < 1/\Delta\nu$. Usando el resultado anterior y $c=\lambda\nu$, podemos relacionarlo con la longitud de onda
	
	\begin{equation}
		l_C = c\tau_C = c\Delta\tau = \frac{c}{\Delta\nu} \implies l_C = \frac{\bar{\lambda}^2}{\Delta\lambda}
	\end{equation}
	
	\begin{Socrative}
		Si $\Delta = 3$cm y tenemos $l_C = 1.5$cm, ¿qué ocurrirá? $\implies$ Sol: Se ve bien la periferia, pero el centro se verá mal.
	\end{Socrative}
	
	\subsection{Coherencia espacial y área de coherencia.}
	
	En el apartado anterior, considerábamos fuentes casi-puntuales pero no monocromáticas, con lo cual
	estudiábamos algunos aspectos de la coherencia temporal.En este,vamos a ponernos en la otra situación
	extrema: fuentes cuasi-monocromáticas pero extensas. (\cite{Resumen})
	
	\esp
	
	Nos ocupamos ahora de definir el área subtendida desde el centro de la fuente que va a permitir que se forme patrón interferencial. Si los dos orificios $P_1$ y $P_2$ están dentro de este área, entonces se
	podrá producir el patrón interferencial. (\cite{Resumen})
	
	\esp
	
	Sea $\Delta A$ el área que contiene $P_1$ y $P_2$, que es aproximadamente $(R\Delta\theta)^2$, donde $\Delta\theta$ es el ángulo sólido. Al igual que antes, se ha obtenido experimentalmente que para que existan interferencias necesitamos que $\Delta\theta \Delta s \leq \bar{\lambda}$, de esta forma obtenemos
	
	\begin{equation}
		\Delta A \approx (R\Delta\theta)^2 \approx \frac{R^2\bar{\lambda}^2}{(\Delta s)^2}
	\end{equation}
	
	\textit{R} sería la distancia desde el centro \textit{S} de $\sigma$ y cada fuente secundaria. A esta área ($\Delta A$) sobre el plano de los orificios denominamos \textbf{área de coherencia}, y a su longitud ($C_T=\sqrt{\Delta A}$) llamamos coherencia transversal, recordemos que a la longitud de coherencia la llamábamos \textbf{coherencia longitudinal}. (\cite{Resumen})
	
	\esp
	
	La coherencia transversal recibe su nombre del hecho de que representa un segmento de longitud dada perpendicular o transversal a la dirección de propagación de la luz que proviene de la fuente. (\cite{Resumen})
	
	\section{Caracterización estadística de las perturbaciones luminosas.}
	
	\subsection{Función de autocorrelación: ejemplo}
	
	Consiste en el área encerrado entre dos copias de la misma función en diferentes posiciones.
	
	\begin{figure}[H]
		\centering
		\includegraphics[width=0.7\textwidth]{IMG/autocorrelacionej.png}
	\end{figure}
	
	\printbibliography[title={Bibliografía}]
\end{document}