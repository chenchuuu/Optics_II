\documentclass[12pt,a4paper]{report}

% --- Paquetes de Idioma y Codificación ---
\usepackage[utf8]{inputenc}
\usepackage[T1]{fontenc}
\usepackage[spanish, es-tabla]{babel}

% --- Diseño de Página ---
\usepackage{geometry}
\geometry{top=2.5cm, bottom=2.5cm, left=3cm, right=3cm}

% --- Matemáticas y Símbolos ---
\usepackage{amsmath, amssymb, amsthm}

% --- Colores y Cajas ---
\usepackage[dvipsnames]{xcolor}
\usepackage[most]{tcolorbox}

% Configuración de cajas personalizadas
\newtcolorbox{importante}{
	colback=red!5!white,
	colframe=red!75!black,
	title=Importante,
	fonttitle=\bfseries
}

\newtcolorbox{nota}{
	colback=blue!5!white,
	colframe=blue!75!black,
	title=Nota,
	fonttitle=\bfseries
}

% Definición de la caja de bibliografía
\newtcolorbox{bibliografia}{
	colback=gray!5!white,
	colframe=gray!75!black,
	title=Bibliografía / Referencias,
	fonttitle=\bfseries,
	breakable
}

% --- Hipervínculos ---
\usepackage{hyperref}
\hypersetup{
	colorlinks=true,
	linkcolor=blue,
	filecolor=magenta,      
	urlcolor=cyan,
}

% --- Datos del Documento ---
\title{Óptica II}
\author{Chengyu Jin}
\date{\today}

\begin{document}
	
	\maketitle
	
	\tableofcontents
	\newpage
	
	\chapter{Teoría de la coherencia parcial de la luz. Aplicaciones.}
	
	\begin{bibliografia}
		\begin{enumerate}
			\item Nieves, J.L., Jiménez J.R. y Hernández Andrés, J. “Introducción a la teoría
			difraccional de la formación de imágenes”. Cap 5.
			\item  Pedrotti, F.L., Pedrotti, L.S. “Introduction to Optics”. Cap 12.
			\item Casas, J. “Óptica”. Cap. 12.
			\item Fowles, G.R. “Introduction to modern optics”. Cap. 3
			\item Born, M. y Wolf, E. “Principles of Optics”. Cap. 10.
			\item Hecht-Zajac.“Óptica”.Cap.12.
		\end{enumerate}
	\end{bibliografia}
	
	\section{Repaso de conceptos elementales y definiciones}
	
	\subsection{Coherencia temporal, tiempo de coherencia y longitud de coherencia.}
	
	Supongamos ahora que tenemos un interferómetro Michelson iluminado con una fuente casi puntual
	$\sigma$ que emite con un ancho de banda en frecuencias $\Delta\nu$.
	
	\begin{nota}
		En la realidad no existen rayos monocromáticos debido al principio de incertidumbre de Heisenberg, los rayos siempre tienen un ancho de banda $\Delta\nu$
	\end{nota}
	
	A la diferencia de recorrido máxima en distancias entre los dos haces para que haya interferencias,
	que se calcula como $c\Delta\tau$, denominamos \textit{longitud de coherencia} de la fuente.
	
	\begin{equation}
		\Delta = 2(d_2-d_1)\cos\theta = c\Delta\tau = l_C
	\end{equation}
	
	Experimentalmente se ha determinado que para que existan interferencias necesitamos que 
	
	\begin{equation}
		\Delta\nu\Delta\tau < 1
	\end{equation}
	
	\noindent esto es que
	
	\begin{equation}
		\Delta\nu = \tau_C < \frac{1}{\Delta\nu}
		\label{eq:coherencia}
	\end{equation}
	
	\noindent Usando la ecuación (\ref{eq:coherencia}) y $c=\lambda\nu$ podemos relacionarlo con la longitud de onda
	
	\begin{equation}
		l_C = c\tau_C = c\Delta\tau = \frac{c}{\Delta\nu} \implies l_C = \frac{\bar{\lambda}^2}{\Delta\lambda}
	\end{equation}
	
	\subsection{Coherencia espacial y área de coherencia.}
	
	En el apartado anterior, considerábamos fuentes casi-puntuales pero no monocromáticas, con lo cual
	estudiábamos algunos aspectos de la coherencia temporal.En este,vamos a ponernos en la otra situación
	extrema: fuentes cuasi-monocromáticas pero extensas.
	
\end{document}